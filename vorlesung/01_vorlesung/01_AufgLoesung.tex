\section{Lösung zur Aufgabe aus Vorl 1}
\subsection{Lösung zum Beispiel zur Methode der der kleinsten Quadrate}
aus Abschnitt \ref{Kap1LMSbeispiel}

Zu Bestimmen ist ein Ohmscher Widerstand $R$ sowie eine Offsetspannung $U_0$ bei gegebenen Werten einer
Präzisionsstromquelle und eines Voltmeters:

\begin{center}
\begin{tabular}{l||c|c|c|c|c|c|c|c|c}
\hline\hline
 $I$ in mA &    4.0 &     6.0 &     8.0 &    10.0 &    12.0 &    14.0 &    16.0 &    18.0 &    20.0\\
\hline
 $U$ in mV &    62.5 &    51.5 &    96.0 &   140.2 &   138.9 &   195.1 &   225.8 &   207.8 &   223.7 \\
\hline\hline
\end{tabular}
\end{center}

Es wird angenommen, dass die Stromstärken ohne Streuung vorliegen (als Regressoren) und die Spannungen normalverteilt
streuen (als Regressanden), so dass
\begin{equation*}
\min_{R, U_0} \; \sum_{i = 1}^J \, \left(U_i \, - \, R \, I_i  \, - \,  U_0\right)^2
\end{equation*}

\begin{itemize}
\item[(a)] Berechnen Sie den Korrelationskoeffizienten $\rho(I, U)$

$$
\rho = \frac{\mathrm{Cov}(U,I)}{\sqrt{\mathrm{Var}(U)} \, \sqrt{\mathrm{Var}(I)}}
$$
Für die Ko-Varianzen werden die Mittelwerte gebraucht und die Anzahl der Wertepaare $J = 9$
$$
\bar U = \frac{1}{J} \sum\limits_{j=1}^J  U_j =
$$
$$
=  (62.5 +   51.5 +   96.0 + 140.2 + 138.9 + 195.1 + 225.8 + 207.8 + 223.7)/9 = 149.06 \, \mathrm{m V}
$$
$$
\bar I = \frac{1}{J} \sum\limits_{j=1}^J I_j =
$$
$$
= (4.0 +  6.0 + 8.0 + 10.0 + 12.0 + 14.0 + 16.0 + 18.0 + 20.0)/9 = 12  \, \mathrm{m A}
$$

$$
\mathrm{Cov}(U,I) = \frac{1}{J-1} \sum\limits_{j=1}^J (U_i - \bar U) (I_i - \bar I) =
$$
$$
= \frac{1}{8}\left( (62.5 - 149.06) (4-12) + \dots + (223.7 - 149.06) (20 - 12) \right)
= 357.05 \, \mathrm{m V} \cdot \mathrm{m A}
$$
$$
\mathrm{Var}(U) = \frac{1}{J-1} \sum\limits_{j=1}^J (U_i - \bar U)^2 = 4629.7 \, (\mathrm{m V})^2
$$
$$
\sqrt{\mathrm{Var}(U)} = 68.042 \, \mathrm{m V}
$$
$$
\mathrm{Var}(I) = \frac{1}{J-1} \sum\limits_{j=1}^J (I_i - \bar I)^2 = 30 \, (\mathrm{m A})^2
$$
$$
\sqrt{\mathrm{Var}(I)} = 5.4772 \, \mathrm{m A}
$$
also
$$
\rho = \frac{357.05}{68.042 \cdot 5.4772} = 0.958
$$
\item[(b)] Lösen Sie das Gleichungssystem
\begin{equation*}
\left(\begin{array}{c}
\frac{\partial}{\partial R}\\
\frac{\partial}{\partial U_0}
\end{array}\right)
\, \sum_{i = 1}^J \, \left(U_i \, - \, R \, I_i  \, - \,  U_0\right)^2 \; = \;
\left(\begin{array}{c}
0\\
0
\end{array}\right)
\end{equation*}
d.h.
$$
\begin{array}{ccc}
\sum_i 2 (U_i - R I_i - U_0) \cdot (-I_i) & = & 0 \\
\sum_i \underbrace{2 (U_i - R I_i - U_0)}_{\text{aeussere Abl.}} \cdot
 \underbrace{(-1)}_{\text{innere Abl.}} & = & 0
\end{array}
$$
d.h.
$$
\begin{array}{ccc}
\sum_i U_i I_i & = & \sum_i I_i^2 R \; + \;  \sum_i I_i U_0 \\
\sum_i U_i & = & \sum_i I_i R \; + \;  \sum_i U_0
\end{array}
$$
mit $\sum_i U_0 = U_0 \sum_i 1 = J U_0$
\begin{equation*}
\left(\begin{array}{cc}
\sum_i I_i^2 & \sum_i I_i\\
\sum_i I_i & J
\end{array}\right)
\left(\begin{array}{c}
R\\
U_0
\end{array}\right)
 \; = \;
\left(\begin{array}{c}
\sum_i U_i I_i\\
\sum_i U_i
\end{array}\right)
\end{equation*}
Lösen des Gleichungssystems
\begin{equation*}
\left(\begin{array}{c}
R\\
U_0
\end{array}\right)
 \; = \;
 \left(\begin{array}{cc}
 \sum_i I_i^2 & \sum_i I_i\\
 \sum_i I_i & J
 \end{array}\right)^{-1}
\left(\begin{array}{c}
\sum_i U_i I_i\\
\sum_i U_i
\end{array}\right)
\end{equation*}
Für die Inverse einer $2 \time 2$-Matrix gilt
$$
\left(\begin{array}{cc}
a & b\\
c & d
\end{array}\right)^{-1} = \frac{1}{ad - cb}
\left(\begin{array}{cc}
d & -b\\
-c & a
\end{array}\right)
$$
so dass
$$
\left(\begin{array}{c}
R\\
U_0
\end{array}\right)
 \; = \;
 \frac{1}{J (\sum_i I_i^2) - \left( \sum_i I_i \right)^2}
 \left(\begin{array}{cc}
 J & -\sum_i I_i\\
 -\sum_i I_i & \sum_i I_i^2
 \end{array}\right)^{-1}
\left(\begin{array}{c}
\sum_i U_i I_i\\
\sum_i U_i
\end{array}\right)
$$
\item[(c)] Berechnen Sie die aus der Lösung des Gleichungssystems aus (b) gewonnen Schätzwerte
für $R$ und $U_0$.

$$
R = \frac{J(\sum_i U_i I_i) - (\sum_i U_i)(\sum_i I_i)}
{J (\sum_i I_i^2) - \left( \sum_i I_i \right)^2} =
\frac{2.5708 \cdot 10^4 \, \mathrm{m V} \cdot \mathrm{m A}}
{2160 \, (\mathrm{m A})^2} = 11.9 \, \frac{\mathrm{V}}{\mathrm{A}} = 11.9 \, \Omega
$$
$$
U_0 = \frac{(\sum_i I_i^2)(\sum_i U_i) - (\sum_i I_i)(\sum_i U_i I_i)}
{J (\sum_i I_i^2) - \left( \sum_i I_i \right)^2} =
\frac{1.3469 \cdot 10^4 \, (\mathrm{m A})^2 \cdot \mathrm{m V}}
{2160 \, (\mathrm{m A})^2} = 6.24 \, \mathrm{m V}
$$

\end{itemize}
