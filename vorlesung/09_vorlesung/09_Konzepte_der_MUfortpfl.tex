\section{Konzepte der Messunsicherheitsfortpflanzung (M.U.F.)}

% 1 Fortpflanzung für lineare Modelle
% 2 Überdeckungsintervall
% 3 Kombinierte Standardabweichung
%   (Sattherthwaite)
% 4 Erweiterungsfaktor

In der Messtechnik geht es darum, indirekte Messgrößen über Beobachtungen von direkten
Messgrößen und über ein Schätzverfahren auf Basis eines Modells zu ermitteln.
In der ersten Vorlesung hatten wir dazu hervorgehoben, dass zu dem Modell zum einen
die mathematische Beschreibung des physikalischen Sachverhaltes gehört und zum anderen
die statistische Beschreibung von nicht deterministischem Verhalten.
Das nichtdeterministische Verhalten eines Systems wird in vielen Fällen
von nicht vorhersehbaren, kleinen Störeinflüssen und Auflösungsbegrenzungen
beim Beobachtungsvorgang verursacht. Es gibt aber auch Prozesse, bei denen
der physikalische Vorgang, der zu untersuchen ist, selber stochastischen Charakter hat,
wie beispielsweise bei radioaktiven Zerfällen, Streuprozessen atomarer Vorgänge oder bei
biologischen Prozessen. Das nicht deterministische Verhalten, die
Stochastik, eines Prozesses wird mittels entsprechender Wahrscheinlichkeitsverteilungen modelliert.

Ein Messergebnis einer Größe $X$ wird deshalb statistisch mit Wahrscheinlichkeitsverteilungen ausgedrückt
\begin{itemize}
\item in Form einer Wahrscheinlichkeitsdichteverteilung
$$
p(X)
$$
oder
\item in Form der statistischen Momente, Erwartungswert $x$ und Varianz (bzw.\ deren Wurzel $\sigma_X$),
einer Wahrscheinlichkeitsdichteverteilung, zusammen mit einer Wahrscheinlichkeit $1-\alpha$
bzw.\ einem Quantil $k$
$$
x = \int X \, p(X) \operatorname d X, \quad
\sigma_X = \sqrt{\int (X-x)^2 \, p(X) \operatorname d X}, \quad k = P^{-1}(1-\alpha/2).
$$
\end{itemize}
Dreh- und Angelpunkt bei der Quantifizierung von Messgrößen ist die Vorgehensweise, wie ein
Mess\-er\-gebnis durch Modellbildung, Parameterschätzung und Bestimmung der Wahrscheinlichkeitsdichtefunktion
der Parameter gewonnen wird. Ein Messergebnis einer Größe $Y$ oder eines Größenvektors
$\mathbf{Y}$, die bzw.\ der \textsl{explizit} oder \textsl{implizit} von direkten
Messgrößen $\mathbf{X}$ abhängt, soll ausgedrückt werden durch eine
Wahrscheinlichkeitsdichteverteilung $p(Y)$ bzw.\ durch den entsprechenden Erwartungswert $y$ und
die Wurzel aus der Varianz $\sigma_Y$ mit einem Quantil $k$, dem Erweiterungsfaktor, für
eine spezifizierte Wahrscheinlichkeit $1-\alpha$.

Wir kategorisieren die Modelle für indirekte Messgrößen
danach, ob sie univariat oder multivariat und ob sie explizit oder implizit sind:

\begin{center}
{\setlength{\extrarowheight}{4pt}%
\begin{tabular}{l||c|c|}
\arraycolsep=4pt\def\arraystretch{2}
 & univariat: $1$ indirekte Größe & multivariat: $M>1$ indirekte Größen \\[4pt]
\hline\hline
explizit & $Y = f(\mathbf{X})$ & $\mathbf{Y} = \vec f(\mathbf{X})$\\[4pt]
\hline
implizit & $f(Y,\mathbf{X})=0$ & $f(\mathbf{Y},\mathbf{X})=0$ oder
 $\vec f(\mathbf{Y},\mathbf{X})=0$\\[4pt]
\hline
\end{tabular}}
\end{center}

mit $\mathbf{X} = (X_1,\dots,X_N)^\mathsf{T}$ und $\mathbf{Y} = (Y_1,\dots,Y_M)^\mathsf{T}$.

Bei der linearen Regression beispielsweise haben wir, wie wir es in der zweiten Vorlesung
gelernt haben, die Regressoren und Regressanden als
\begin{enumerate}
\item gemeinsam ausgelesene (z.B.\ durch Triggerung zugeordnete) Beobachtungstupel
\begin{equation}
\boldsymbol (Y_{\mathrm{Regr}, j}, \mathbf{X}_j) \; = \; (Y_{\mathrm{Regr}, j}, X_{1,j}, \dots, X_{M,j})
\end{equation}
vorliegen mit $j = 1,\dots, J$ und $J$ für den Stichprobenumfang, so dass gemäß einem
zeitlichen Ablauf eine Veränderung aller Größen vorliegen kann,
\item ansonsten können die direkten Messgrößen in voneinander unabhängigen Experimenten
und damit im allgemeinen auch mit verschiedenen Stichprobenumfängen gewonnen werden.
\end{enumerate}

Der erstere Fall repräsentiert als direkte Größen
sowohl die Regressoren $\mathbf{X}$ als auch die Regressanden $Y_{\mathrm{Regr}}$
und als indirekte Messgrößen, die Modellparameter $\boldsymbol \theta \equiv \boldsymbol Y$.
Hier handelt es sich also hinsichtlich der Unsicherheitsfortpflanzung um ein
implizites, multivariates Problem, auch bei der univariaten, linearen Regression.
\begin{equation}
0 = f(\mathbf{Y},\mathbf{X}) \equiv f(\boldsymbol \theta,\mathbf{X})
= Y_{\mathrm{Regr}} - \sum_{i=1}^{M} \theta_i \, X_i
\label{implizitMultivariatRegre}
\end{equation}
Das \glqq Multivariate\grqq ~für die Unsicherheitsberechnung betrifft die indirekten
Messgrößen $\boldsymbol \theta \equiv \boldsymbol Y$, das \glqq Univariate\grqq ~für die
Regression betrifft den Regressanden $Y_{\mathrm{Regr}}$.

Fall Zwei kann im allgemeinen so geartet sein, dass jede der direkten Größen in jeweils unabhängigen
Messvorgängen gewonnen wird, bei denen im allgemeinen unterschiedliche Stichprobenumfänge $J_i, J_k$ vorliegen
können und dann keine paarweise Zuordnung der einzelnen Beobachtungswerte $X_{i,j}$ zu $X_{k,l}$ für
$i \neq k$ vorliegt.

Bei dem expliziten, univariaten Fall
\begin{equation}
Y = f(\mathbf{X})
\label{explizitUnivariat}
\end{equation}
können die direkten Messgrößen als unkorrelierte unterschiedliche Stichproben vorliegen (Fall 2)
oder auch als gemeinsam getriggerte Tupel
\begin{equation}
\boldsymbol X_j \; = \; (X_{1,j}, \dots, X_{N,j})
\end{equation}
wie in Fall 1, aber dann nicht mit Regressionskoeffizienten als Modellparameter, sondern
mit der indirekten Größe explizit als Funktion von $(X_{1}, \dots, X_{N})$ gemäß
Gl.~(\ref{explizitUnivariat}).


Für eine internationale Vergleichbarkeit und den internationalen Handel ist es erforderlich,
dass die Verfahren zur Bestimmung von Messergebnissen möglichst einheitlich sind.
Für das gesetzliche Messwesen ist deshalb die international übereingekommene Richtlinie
zur Bestimmung von Messunsicherheiten bindend.
Die internationale Richtlinie zur Berechnung von Messunsicherheiten umfasst mehrere
Dokumente:
\begin{itemize}
\item JCGM 100:2008 GUM 1995 with minor corrections \textsl{Evaluation of measurement
data - Guide to the expression of uncertainty in measurement} \cite{GUM95}; oftmals einfach mit \glqq GUM\grqq ~bezeichnet,
betrifft univariate und explizite Modelle,
  \begin{itemize}
  \item die linear sind
  \begin{equation}
  Y = f(\mathbf{X}) = \sum_{i=1}^N c_i X_i
  \label{univarLinear}
  \end{equation}
  oder für die eine Linearisierung zulässig ist
  \begin{equation}
  Y = f(\mathbf{X})
  = \left. f(\mathbf{X}) \right|_{\bar{\mathbf{x}}} +
    \sum_{i=1}^N \underbrace{\left. \frac{\partial f}{\partial X_i} \right|_{\bar{\mathbf{x}}}}_{c_i} \Delta X_i
  \label{univarTaylorLin}
  \end{equation}
  \item die Zufallsgrößen betreffen, deren Streuung (\textsl{dispersion}) normalverteilt oder
  $t$-verteilt ist.
  \end{itemize}
Die indirekte Messgröße $Y$ ist damit Linearkombination der direkten Messgrößen $X_i$, d.h.
$Y = \sum c_i X_i$, so dass für die Varianz $\operatorname{Var}(Y) = \operatorname{Var}(\sum c_i X_i)$
gemäß Gl.~(\ref{univarLinearFortpflanzungKap1}) aus Abschnitt \ref{Kap1Kovarianzen} gilt
\begin{equation}
\operatorname{Var}(Y) = \sum_{i=1}^N \sum_{k=1}^N  c_i c_k \operatorname{Cov}(X_i, X_k)
\label{univarLinearFortpflanzung}
\end{equation}
Da sich viele physikalische Zusammenhänge für definierte Bereiche linearisieren lassen,
findet Gl.~(\ref{univarLinearFortpflanzung}) in großen Teilen der Messdatenanalyse
Anwendung und ist in der Literatur als das \textsl{Gesetz zur Fortpflanzung von Messunsicherheiten}
bekannt.

\item JCGM 101:2008 \textsl{Evaluation of measurement
data - Supplement 1 to the Guide to the expression of
uncertainty in measurement - Propagation of distributions using a Monte Carlo method} \cite{GUMS1};
kurz mit \glqq GUM - supplement 1\grqq ~bezeichnet, betrifft
explizite und implizite, univariate Modelle,
  \begin{itemize}
  \item die sich nicht einfach linearisieren lassen
  \item und/oder die sich nicht in einer geschlossenen analytischen Form darstellen lassen.
  \end{itemize}
Hier wird zu jeder direkten Messgröße eine Wahrscheinlichkeitsdichteverteilung vorgegeben.
Gemäß den jeweiligen Verteilungen wird eine große Stichprobe (ein großes \textsl{Sample})
$$
\mathbf{x}_1 = (x_{1,1},\dots,x_{N,1})^\mathsf{T}, \dots, \mathbf{x}_J = (x_{1,J},\dots,x_{N,J})^\mathsf{T}
$$
für die direkten Größen $\mathbf{X}$ per Zufallszahlengenerator erzeugt. Diese wird in das Modell $f$ gesteckt,
um eine Stichprobe
\begin{equation}
y_1 = f(x_{1,1},\dots,x_{N,1}), \dots, y_J = f(x_{1,J},\dots,x_{N,J})
\end{equation}
der indirekten Größe $Y$ zu gewinnen. Das Histogramm der Stichprobe der indirekten Größe liefert somit die
Wahrscheinlichkeitsdichteverteilung. Aus der sortierten Stichprobe nach der Art wie wir sie
für den Kolmogoroff-Smirnow-Test berechnet haben, gewinnen wir die kumulierte Wahrscheinlichkeitsverteilung
aus deren inverser Funktion das Überdeckungsintervall gewonnen werden kann.\\
$\rightarrow$ Dieses Verfahren wird Kapitel~\ref{montecarloMU} dargelegt.
\item JCGM 102:2011 \textsl{Evaluation of measurement data – Supplement 2 to the
Guide to the expression of uncertainty in measurement - Extension to any number of output quantities}
liefert die Verallgemeinerung der beiden Richtliniendokumente JCGM 100 und JCGM 101 für den multivariaten
Fall mit mehreren $M>1$ indirekten Größen, also einem Größenvektor $\mathbf{Y}$.
Hier wird dargelegt, wie man die Unsicherheit
  \begin{itemize}
  \item bei linearen oder linearisierbaren Modellen
    mittels der Berechnung der Kovarianzen (in der Art wie wir es
    in Kapitel~\ref{KapitellinReg} für die lineare Regression kennen gelernt haben)
  \item oder im Fall der nicht linearisierbaren und/oder komplexeren (oftmals impliziten) Modelle
    via Monte-Carlo-Berechnungen analog zu GUM-supplement 1
  \begin{equation}
  \mathbf{y}_1 = \vec f(x_{1,1},\dots,x_{N,1}), \dots, \mathbf{y}_J = \vec f(x_{1,J},\dots,x_{N,J})
  \end{equation}
  \item sowie für alle Fälle, ob uni- oder multivariat, ob explizit oder implizit, ob analytisch
    oder via Monte-Carlo-Verfahren unter Einbeziehung von {\`a} priori-Information mittels
    bayesischer Methoden\\
    $\rightarrow$ Kapitel~\ref{bayesMU}
  \end{itemize}
ermittelt.
\item JCGM 103 CD 2018-10-04 \textsl{Guide to the expression of uncertainty in measurement
- Developing and using measurement models} behandelt in umfassender Weise die Problematik der
Modellentwicklung. Es gibt den Bereich physikalischer, deterministischer Prozesse,
den Bereich der nicht-deterministischen, physikalischen oder biologischen oder soziologischen
Prozesse. Ferner gibt es den Bereich der statistischen Modelle, die dazu dienen, nicht-deterministische
Anteile von Prozessen zu behandeln. Das JCGM 103 Dokument soll diese Dinge konzeptionell
für die Metrologie beleuchten. Es ist bisher ein Entwurf; das Kürzel CD steht für \textsl{committee draft}.
\item JCGM 104:2009 \textsl{Guide to the expression of uncertainty in measurement
- An introduction to the Guide to the expression of
uncertainty in measurement and related documents} liefert die Konzepte und Hintergründe zur
Unsicherheitsbestimmung. Dieses Dokument soll ein Verständnis für die Konzepte der
Wahrscheinlichkeitsdichteverteilungen im Zusammenhang mit der Bestimmung von Messunsicherheiten liefern.
\item JCGM 106:2012 \textsl{Evaluation of measurement data - The role of
measurement uncertainty in conformity assessment}. Im Rahmen der Vorlesungen zu Hypothesentests und
Ringvergleichen haben wir bereits gelernt, dass Messergebnisse zu vergleichen sind, und wie dies
gehandhabt wird. In der Qualitätssicherung ist die zentrale Aufgabe, gefertigte Bauteile mit den
vorgegebenen Daten der technischen Konstruktionszeichnung zu vergleichen hinsichtlich der Übereinstimmung,
also der Einhaltung von Toleranzgrenzen. Dies wird Konformitätsbewertung genannt und wird im
JCGM 106-Dokument behandelt.
\end{itemize}

Das historisch älteste Dokument JCGM~100 gilt für explizite, univariate, skalare und
gauß- oder $t$-verteilte indirekte Messgrößen mit linearem oder linearisierbarem Modell für die
Abhängigkeit der indirekten Messgröße von den direkten Messgrößen. Es wurde zunächst erweitert durch das JCGM~101. Die
in JCGM~101 spezifizierte Monte-Carlo-Methode ermöglicht es, dass man nicht auf die Bestimmung der Unsicherheit
expliziter, univarter indirekter Messgrößen mit linearisierbarem Modell begrenzt ist.
Es bietet die Möglichkeit auch die Unsicherheit für implizite, multivariate Messgrößen,
deren direkte Eingangsgrößen beliebig verteilt sein können zu ermitteln. Die Verteilung der
Größen können eine von Null verschiedene Skewness aufweisen. Sie können sogar die Form einer
U-Verteilung haben, was beispielsweise bei direkten Größen, die durch Vibrationen beeinflusst
werden, vorkommen kann. Die Größen können irgendwelchen Verteilungen gehorchen, so dass die
Wahrscheinlichkeitsverteilungen der indirekten Messgrößen entsprechend einen
irgendwie gearteten Verlauf aufweisen können.

Die nächste Erweiterung liefert dann das JCGM~102-Dokument, das eine vollständige Verallgemeinerung
darstellt. Demgemäß werden auch Verfahren betrachtet, die die Berücksichtigung der Unsicherheit des Modells an sich
zulassen (die indirekte Messgröße aufgrund von Mangel an Information als Zufallsgröße behandelt)
sowie die Behandlung von vorherigen Informationen über die indirekten Messgrößen (Bayesische Statistik).

Die Dokumente JCGM 100, 101, 102, 104 und 106 sind über die Webseiten des
\textsl{Bureau International des Poids et Mesure}, abgekürzt BIPM,
\begin{verbatim}
https://www.bipm.org/en/publications/guides
\end{verbatim}
erhältlich.

Die wesentlichen Komponenten der Messunsicherheit sind
\begin{itemize}
\item die Begrenztheit des Messvorgangs bedingt durch
  \begin{itemize}
  \item endliche Auflösung der Geräte,
  \item Messbereichsgrenzen,
  \item Störeinflüsse von außerhalb und innerhalb der Geräte,
  \item subjektive Komponenten durch den Operateur,
  \end{itemize}
\item die Begrenztheit der Modellbildung bedingt durch
  \begin{itemize}
  \item Vereinfachungen (Parsimonie = \glqq Sparsamkeit\grqq)
    und Grenzen durch Rechenkapazität zeitlich (Rechenzeit/Rechenleistung)
    und kostenmäßig (Speicherkapazität und Maschinengenauigkeit,
    numerische Stabilität, numerische Zuverlässigkeit),
  \item Vereinfachungen mangels Kenntniss über quantitative Details
    zu Einflussgrößen
    oder genaueren Details der Physik innerhalb des
    betreffenden physikalischen Effektes,
  \item mangels Informationen zu genaueren Werten von Einflussgrößen der
    physikalischen Prozesse.
  \end{itemize}
\end{itemize}

Bei der Schätzung von Modellparametern (Quantifizierung der indirekten Messgrößen)
spielt vielfach die numerische Zuverlässigkeit des Optimierungsalgorithmus eine
wichtige Rolle. Wenn wir das Kosten\-funktions\-beispiel der Abbn.~\ref{LSoptiExample1NM} und \ref{LSoptiExample1Grad}
mit dem der Abb.~\ref{LSoptiExample1NM} in Kapitel~\ref{LSoptiExample1SinusRQS} vergleichen,
erkennen wir, dass die Modellparameter bei Abb.~\ref{LSoptiExample1NM} und \ref{LSoptiExample1Grad}
in etwa die gleiche Skalierung aufweisen, während die in Abb.~\ref{LSoptiExample1NM} stark unterschiedlich sind. In Abb.~\ref{LSoptiExample1NM} und \ref{LSoptiExample1Grad} sehen wir eine schöne, runde
Kuhle und in Abb.~\ref{LSoptiExample1Grad} einen schmalen Graben. Im letzteren ist das Modell mit seinen
Parametern nicht so gut konditioniert, so dass das Minimum nicht klar erkennbar ist.
Nicht so gut konditionierte Modelle führen eine größere Unsicherheit bei der Schätzung
seiner Parameter mit sich.

Für die Wahl des Modellansatzes gilt es nicht nur den physikalischen Zusammenhang
geeignet wieder zugeben, sondern auch das Verhalten der Gleitkommaarithmetik zu
berücksichtigen. Wenn zwei sehr große Zahlen voneinander zu subtrahieren sind, muss
berücksichtigt werden, wie hoch die Maschinengenauigkeit ist, also wie groß die
Mantisse ist. Werden die Zahlen so groß, dass die relevanten Nachkommastellen, die
nach Subtraktion gewonnen werden sollen, nicht vorhanden sind, liefert der Algorithmus
kein brauchbares Ergebnis.
\begin{quote}
Doing floating point operations is like moving piles of sand: every
time you move a pile, you lose a little sand and pick up a little dirt!
\end{quote}
Ein Algorithmus ist deshalb derart zu implementieren, dass hohe Potenzen oder
vielfache Multiplikationen vor Additionen oder Subtraktionen zu vermeiden, zu umgehen
sind. Dazu gehört es, zu überlegen, wie man das Modell formuliert, beispielsweise
ob man Polynome in der einfachen Darstellung, als Legendre oder als Tschebycheff-Polynome
definiert. Dazu gehört auch, zu prüfen, ob man und wie man ein Modell zunächst in einer
einfacheren Approximation aufbaut und erforderlichenfalls Effekte höherer Ordnung in
einem nächsten Schritt hinzufügt.

Ein Modell ist so zu wählen, dass es der entsprechenden Anwendung gerecht wird
hinsichtlich der Anforderungen an Kosten vs.\ Genauigkeit. Wie gut ein Modell einen
physikalischen Sachverhalt beschreibt, wird in der Modellunsicherheit ausgedrückt.
Ein kleines Beispiel hatten wir in Kapitel~\ref{KapitellinReg} gesehen, in der ein Vergleich
zwischen einer Regressiongeraden und einem Polynom 6.\ Grades gemacht wurde, siehe
dort das in Abb.~\ref{fig:LineareRegressionMoeglFits} dargestellte Beispiel.
